\chapter*{Abstract}

The solar cycle gives rise to complex structures and dynamics on the Sun. As a result of magnetic disturbances on the Sun, large bursts of energy lead to space weather events that can be potentially harmful to life on Earth. In this thesis, we present a series of projects exploring the themes of \gls{cr} space weather applications and understanding the solar interior-atmosphere linkage. 
%Solar activity varies periodically, which is known as t

A feasibility study was performed to determine whether the \gls{hisparc} network was suitable for detecting space weather events. Using simulations and \gls{hisparc} data, we found few events would be detectable in the nominal \gls{hisparc} configuration, aside from the most extreme events, hence we explored an alternative configuration. We introduced a new configuration of \gls{hisparc} station, which minimised \gls{cr} energy biases and noise. This configuration was demonstrated to significantly improve the capabilities of \gls{hisparc} as a space weather monitor and we provided compelling evidence to suggest that we should be capable of observing \glspl{gle}. In addition, we showed the sensitivity would be further enhanced by using a network of detectors.
%properties and data of five \gls{hisparc} stations were examined in detail, and we showed there were no clear observations of \glspl{gle} or \glspl{fd} in the data. \gls{as} simulations were performed to further explore the response of \gls{hisparc} and our predictions suggested that variations from \glspl{gle} would be minimal.
%demonstrated the performance of this station, including the mean count rate, noise, and response to \glspl{gle}, to demonstrate its capabilities as an improved space weather monitor.

Analysing long-term variations of \glspl{gcr} versus \gls{ssn} during recent solar cycles, we explored the relationship between solar activity and \glspl{gcr}. Focussing on the most recent Cycle 24, we showed it behaved in-accordance with previous even-numbered cycles.
%A study of the l


Finally, a frequency-domain analysis of over 20~years of high-cadence \gls{bison} observations of the \gls{smmf} was presented. This provided evidence to suggest the strongest component of the \gls{smmf} is connected to \glspl{ar}, based both on the inferred lifetime and location on the solar disc. We also searched for evidence of a magnetic signature of global Rossby modes ($r$ modes) in the residual spectrum of the \gls{smmf}, where we did not conclusively find an $r$ mode signal in the data.