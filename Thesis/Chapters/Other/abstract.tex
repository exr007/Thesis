\chapter*{Abstract}

Solar activity varies periodically, which is known as the solar cycle. The solar cycle is driven by the Sun's dynamo processes, giving rise to complex structures and dynamics in the outer layers of the Sun. As a result of the solar activity, magnetic disturbances on the Sun lead to large bursts of energy release which caused space weather effects, which can be potentially harmful to civilisation.

[solar activity / space weather importance...]


In this thesis, we present a series of results that explore the themes of understanding the solar interior-atmosphere linkage and space weather applications.
%In this thesis, a number of projects are presented which explore the themes of understanding the solar interior-atmosphere linkage and space weather applications. This is broken down into three major projects: a feasibility study on \gls{cr} space weather applications, an investigation into the effects of solar activity on \gls{cr} observations, and a study of the \gls{smmf}.

A feasibility study was performed to determine whether the \gls{hisparc} network is suitable for monitoring space weather events. We explored the properties of five \gls{hisparc} stations in detail and analysed their data, showing that there were no clear observations of \glspl{gle} or \glspl{fd}. \gls{as} simulations were performed to further explore the expected response for the \gls{hisparc} stations where our evidence suggested that typically variations from \glspl{gle} would be minimal.

We introduce a new configuration of \gls{hisparc} station, which minimises source of additional noise. We show the performance of this station, including the mean count rate, noise, and perform simulations of \glspl{gle} to demonstrate its performance as an improved space weather monitor.

A study of the long-term variations of \glspl{gcr} versus \gls{ssn} during the most recent solar cycles explored the relationship between solar activity and \glspl{cr}...

...

We present a frequency-domain analysis of over 20~years of high-cadence \gls{bison} observations of the \gls{smmf}. This provides evidence to suggest the strongest component of the \gls{smmf} is connected to \glspl{mfc} and \glspl{ar}, based both on its inferred lifetime and location on the solar disc.