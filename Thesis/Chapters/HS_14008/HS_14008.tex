\chapter{HiSPARC Station 14008}\label{chap:HiSPARC_14008}

%%%%%%%%%%%%%%%%%%%%%%%%%%%%%%%%%%%%%%%%%%%%%%%%%%%%%%%%%%%%%%%%%%%%%
%%%%%%%%%%%%%%%%%%%%%%%%%%%%%%%%%%%%%%%%%%%%%%%%%%%%%%%%%%%%%%%%%%%%%
\section{Introduction}\label{sec:HS_14008_intro}


... [on daily variations (DV)] Dr. Rolf Butikofer (in a reply from Danislav Sapundjiev, dasapund@meteo.be) said:

\textit{"The daily cosmic ray variation near Earth is caused by the anisotropy of the cosmic ray intensity in the interplanetary space. Cosmic ray particles follow the field lines of the interplanetary magnetic field when they travel towards the interior of the heliosphere. Because of the rotation of the Earth, the angle between the asymptotic cone of acceptance of various energies at the location of ground-based cosmic ray detectors (neutron monitors) and the direction of the interplanetary magnetic field varies with a time period of 24 hours. As a consequence cosmic ray detectors look in different directions in the course of a day and observe therefore a diurnal variation. The daily variations of neutron monitors is mainly seen by high latitude stations which have asymptotic directions at low energies (rigidities) near the equator."}



It was clear from the work covered in Chapter~\ref{chap:HiSPARC}, using data from the HiSPARC network, that the HiSPARC detectors were not clearly capable of observing space weather events and this is also hindered as they are rather sensitive to variations in the terrestrial conditions. 

To some extent, it was possible to eliminate the variation in CRs due to terrestrial variation from the HiSPARC data; however it was shown to be not always so simple, as different detectors in the HiSPARC network showed different responses to pressure and temperature variation. The non-linear relationship between temperature and CR count means the correction of the count rate due to thermal fluctuations is non-trivial, unlike the counterpart correction for pressure. 

It is believed that the atmospheric thermal fluctuations induce thermal noise in the \glspl{pmt}, and although the temperature inside the HiSPARC roof boxes have not been measured, it is suspected that the \glspl{pmt} can get quite hot, in particular when the sky boxes are in direct sunlight.

An instance of thermal noise in a single \gls{pmt} will be random, and uncorrelated with an instance of thermal noise in another \gls{pmt}. It is therefore possible to hypothesise that it is unlikely that within the coincidence window of $\sim$1.5 $\mu \mathrm{s}$, that a coincidence between 2 \glspl{pmt} would be due to random thermal noise induced in the \glspl{pmt}.

To exploit this, it is possible to stack 2 detectors on top of each other to measure a single muon which traverses both scintillators, hence inducing signals in both \glspl{pmt}.



%%%%%%%%%%%%%%%%%%%%%%%%%%%%%%%%%%%%%%%%%%%%%%%%%%%%%%%%%%%%%%%%%%%%%
%%%%%%%%%%%%%%%%%%%%%%%%%%%%%%%%%%%%%%%%%%%%%%%%%%%%%%%%%%%%%%%%%%%%%
\section{Aims}\label{sec:HS_14008_aims}

The aim of creating a new HiSPARC station was to test whether an alternative configuration of HiSPARC station could minimise atmospheric deviations in the data and allow for the observation of space weather events...

... provide a more robust detector for observing muons ...



%%%%%%%%%%%%%%%%%%%%%%%%%%%%%%%%%%%%%%%%%%%%%%%%%%%%%%%%%%%%%%%%%%%%%
%%%%%%%%%%%%%%%%%%%%%%%%%%%%%%%%%%%%%%%%%%%%%%%%%%%%%%%%%%%%%%%%%%%%%
\section{HiSPARC 14008 Detector Set-up}\label{sec:HiSPARC_14008}


%%%%%%%%%%%%%%%%%%%%%%%%%%%%%%%%%%%%%%%%%%%%%%%%%%%%%%%%%%%%%%%%%%%%%
\subsection{Configuration}

The configuration of HiSPARC station 14008 is shown in Figure \ref{fig:14008_config}; the station is composed of two scintillators stacked on top of each other, inside one roof box. 

Each of the plastic scintillators has a thickness of $\Delta x \, = \, 0.5$~cm, and density, $\rho \, = \, TBC$~gcm$^{-3}$ . We know that the stopping power of the scintillator for a minimum ionising particle is... % $\sim 2$~MeV/cm$^{-2}$ [REF]... [discuss this w/ Angela as it seems to be derivative of Blethe-Bloch, but cannot find link]...

\begin{figure}
	\center
	\includegraphics[width=0.5\columnwidth]{14008_config.png}
	\caption{Schematic diagram of the HiSPARC station 14008 detector set-up.}
	\label{fig:14008_config}
\end{figure}

\begin{equation}
E = \Delta x \, S \, \rho \, \cos(\theta)
\label{eq:energy_loss}
\end{equation}

\cite{bartels_hisparc_2012} state that typical energy loss of a muon in a single scintillator is $3.38$~MeV, hence in this configuration, as a muon traverses two scintillators, the lower limit on the energy loss by muons in the detector is $\sim 6.76$~MeV.

To protect the scintillators and \glspl{pmt} within the roof boxes, we sandwiched the scintillators between layers of foam, as can be seen on the lab work bench in Figure~\ref{fig:14008_detectors}. Upon complete assembly of the detector, the scintillators and \glspl{pmt} are placed within the roof box on the roof of the Poynting Physics building on the campus of University of Birmingham, as shown in Figure~\ref{fig:14008_ski_box}.

\begin{figure}[ht]
	\centering
	\subfloat[Scintillators on the work bench]{
		\includegraphics[width=0.48\columnwidth]{detectors.jpg}
		\label{fig:14008_detectors}}
	%\qquad
	\subfloat[Complete detector on the roof]{
		\includegraphics[width=0.48\columnwidth]{ski_box.jpg}
		\label{fig:14008_ski_box}} \\
	
	\caption{HiSPARC 14008 assembly and configuration. (a) shows the stacked arrangement of the scintillators on the lab work bench, between layers of pretective foam. (b) shows the complete detector inside the roof box on the University of Birmingham campus.}
	\label{fig:HS_14008_setup}
\end{figure}


%%%%%%%%%%%%%%%%%%%%%%%%%%%%%%%%%%%%%%%%%%%%%%%%%%%%%%%%%%%%%%%%%%%%%
\subsection{Calibration}

When setting up the HiSPARC station, it was required to set several operating parameters for the detectors and the HiSPARC electronics box. One such setting was the \gls{pmt} operating voltage. Each of the detector \glspl{pmt} needs to be powered with a high enough operating voltage such to provide an amplified signal, but not too high such as to over-amplify the noise.

In general, the \glspl{pmt} has an advised operating voltage of around 700~V \citep{fokkema_hisparc_2019}; however, best practise is to operate the \gls{pmt} at the plateu region, whereby the counts/voltage no longer increases. As can be seen from Figure~\ref{fig:PMT_cal}, neither of the \glspl{pmt} have clear plateau regions, hence there was no obvious \gls{pmt} set point.

The HiSPARC installation manual does, however, suggest to tune the \gls{pmt} voltages such that the singles rates for each detector meet the following criteria: singles rate of 100--130 Hz for signal above the high trigger threshold, and singles rate of $<$400 Hz for signal above the low trigger threshold \citep{fokkema_hisparc_2019}.

In order to calibrate the \glspl{pmt} to the correct level, we measured the singles rates above the high and low thresholds as a function of \gls{pmt} operating voltage, as is shown in Figure~\ref{fig:PMT_cal} [UPDATE THIS PLOT...!!!!]. The voltage calibration plot shows drastically the different performances one can get from different \glspl{pmt}, therefore it is necessary to treat each \gls{pmt} individually when calibrating.

\begin{figure}
	\centering
	\includegraphics[width=0.75\columnwidth]{both_PMTs_post_NIM.png}
	\caption{Voltage calibration curve for the PMTs of station 14008. The upper, red-dashed line indicates the upper limit for the low threshold singles rate (400 Hz), and the lower 2, black-dashed lines indicate the upper and lower bounds for the high threshold singles rate (100--130 Hz).}
	\label{fig:PMT_cal}
\end{figure}




%%%%%%%%%%%%%%%%%%%%%%%%%%%%%%%%%%%%%%%%%%%%%%%%%%%%%%%%%%%%%%%%%%%%%
\subsection{Monitoring Temperature}

In Chapter~\ref{chap:HiSPARC}, we suspected that the singles count rates (and thus event count rates also) were affected by the temperature of the \gls{pmt} within the HiSPARC roof-boxes.

Some of the existing HiSPARC stations monitor local temperature however none measure the temperature of the \gls{pmt} within the roof box; therefore the temperature of the PMT itself is unknown, and thus we cannot account for the thermal noise. When building this new HiSPARC station, a temperature sensor was placed into the roof box which allowed us to monitor the temperature.

Figure~\ref{fig:temperature_sensor_circuit} shows the schematic for the temperature sensor. We used the DS18B20 temperature sensor with the one-wire telemetry protocol, which used a single wire to transmit the temperature readings to the microcontroller; the microcontroller used was a Raspberry Pi 4 (see Section~\ref{sec:HS14008_data_acqusition}. The temperature is read on a 10-second cadence and is recorded in degrees Celsius. Three wires were used for the operation of the DS18B20: constant current voltage, ground, and data.

\begin{figure}
	\centering
	\includegraphics[width=0.6\columnwidth]{HS_14008_temp_circuit.png}
	\caption{Schematic diagram of the DS18B20 temperature sensor circuit, whereby the voltage, ground, and GPIO interfaces connect directly into pins of the Raspberry Pi board.}
	\label{fig:temperature_sensor_circuit}
\end{figure}




%%%%%%%%%%%%%%%%%%%%%%%%%%%%%%%%%%%%%%%%%%%%%%%%%%%%%%%%%%%%%%%%%%%%%
\subsection{Data Acquisition}
\label{sec:HS14008_data_acqusition}

The reduced coincidences data is acquired using the typical HiSPARC data acquisition software, but the full coincidences, reduced coincidences, and the temperature data are all acquired by a Raspberry Pi 4.

[show config.]

We use the Raspberry Pi to control the data acquisition by running a Python program to output the coindidences data from the NIM crate and the temperature readings from the sensor to local files.

Each new day generates a separate file for the coincidences data ... 

\begin{table}[ht!]
	\begin{center}
		\caption{Variables stored in the coincidences files of the HiSPARC 14008 instrument.}
		\label{tab:HS_14008_coincidences_data}
		\begin{tabular}{c l c c}
			\hline 
			{\bf Column} & {\bf Item} & {\bf Unit} & {\bf Type} \\ 
			\hline 
			\multirow{2}*{0} & \multirow{2}*{Time Stamp} & YYYY\_MM\_DD & \multirow{2}*{String}  \\ 
			  &  & HH:MM:SS.ffffff & \\ 
			1 & Time*  & Decisecond & Integer, eight digits, zero padded \\ 
			2 & Cumulative Reduced Count* & Counts & Integer, eight digits, zero padded \\ 
			3 & Cumulative Full Count* & Counts & Integer, eight digits, zero padded \\ 
			\hline 
		\end{tabular} 
	\end{center}
	* Since restart
\end{table}

Within the temperature file, there is not header and the data begins from line 1. The columns in the data file are outlined in Table~\ref{tab:HS_14008_temperature_data}.

\begin{table}[ht!]
	\begin{center}
		\caption{Variables stored in the temperature files of the HiSPARC 14008 instrument.}
		\label{tab:HS_14008_temperature_data}
		\begin{tabular}{c l c c}
			\hline 
			{\bf Column} & {\bf Item} & {\bf Unit} & {\bf Type} \\ 
			\hline 
			\multirow{2}*{0} & \multirow{2}*{Time Stamp} & YYYY\_MM\_DD & \multirow{2}*{String}  \\ 
			  &  & HH:MM:SS.ffffff & \\ 
			1 & Temperature & $^\circ$C & Floating point \\ 
			\hline 
		\end{tabular} 
	\end{center}
\end{table}


The daa is stored locally, but is also stored on the University of Birmingham particle physics servers\footnote{Disk location: /disk/moose/general/epesv001/datadisk/147.188.46.117\_hisparc\_pi/}.

%%%%%%%%%%%%%%%%%%%%%%%%%%%%%%%%%%%%%%%%%%%%%%%%%%%%%%%%%%%%%%%%%%%%%
\subsection{Monitoring Pressure}

As with the previous chapter, it was still necessary to account for the barometric effect on the muon count rate. To monitor the pressure, a nearby station was used, which is part of the \gls{midas} database, and acquired from the \gls{stfc} and \gls{nerc} \gls{ceda} archive.

The station used is the nearest \gls{midas} station, and provides a robust measure of the local atmospheric pressure as measured at the station level - a correction for altitude is not applied. The station is located in Coleshill, Warickshire (ID: 19187), nearby Birmingham International Airport, $\sim 20$~km from the HiSPARC detectors, but we believe the pressure variation is small over this distance.

The pressure is read on a 1-hour cadence and is recorded in units of hPa with a precision of 0.1~hPa. The time variation of pressure is slow; hence, we linearly interpolated the data to provide a 1-minute sample.



%%%%%%%%%%%%%%%%%%%%%%%%%%%%%%%%%%%%%%%%%%%%%%%%%%%%%%%%%%%%%%%%%%%%%
%%%%%%%%%%%%%%%%%%%%%%%%%%%%%%%%%%%%%%%%%%%%%%%%%%%%%%%%%%%%%%%%%%%%%
\section{Atmospheric Corrections}\label{sec:HS_14008_atmospheric_correction}






%%%%%%%%%%%%%%%%%%%%%%%%%%%%%%%%%%%%%%%%%%%%%%%%%%%%%%%%%%%%%%%%%%%%%
%%%%%%%%%%%%%%%%%%%%%%%%%%%%%%%%%%%%%%%%%%%%%%%%%%%%%%%%%%%%%%%%%%%%%
\section{Observations}\label{sec:HS_14008_observations}

From the \gls{corsika} simultions in the earlier section, we predict a ground level muon rate passing through the detectors...

(do this by intergrating under curve, using 70-degre half-angle cone for solid angle, and area of 0.5m2)
% i.e. in python doing doing:
% where df contains the alpha and proton diff fluxes
% v = scipy.integrate.simps(df_a_v[1]+df_p_v[1], df_a_v.index.values)
% sr = 4*np.pi*(np.sin(np.deg2rad(70/2)))**2
% area = 0.5
% rate_v = v*sr*area

Get the rate as: $85.365 \, \mu/\mathrm{s}$ (non-vertical, i.e. 70-deg sims) or $156.924 \, \mu/\mathrm{s}$ (for vertical sims)

... for a typical day with this station, we have a count rate of $\sim 80 \, \mu/\mathrm{s}$...


[now move onto the pressure and temperature corrected data and discuss the observations that we see - do we need to have a separate pressure and temperature correction section???]

[Note: temperature correction doesn't work/not necessary on the coincidences but it is necessary on the singles rates!]


[What is the width of the signals generated by the NIM crate? Is it more or less than the approx. 25ns FWHM of the pulses..? Then relate that to: "The pulse width Tw is important only insofar as it determines the maximum rate of pulses that may be represented by the pulse train, since pulses which occur more frequently than 1/Tw cannot be resolved"...]


Investigated the random noise which is induced by random/spurious counts between both PMTs that do not coincide with the passage of a muon. This was achieved by adding a delay between the two PMTs, to ensure any coincident triggers were not due to the passage of a single muon. A delay of over 100~ns was added between the two PMT signals, as the FWHM of a typical pulse is $\sim$~25~ns, and the total duration is on the order of 100~ns.

The delay was added between the two PMTs for around a week and the time series of the coincidences are shown in Figure~\ref{fig:random_coinciences}. We can see that the noise is roughly +/-~1~count/minute...

\begin{figure}[ht!]
	\centering
	\includegraphics[width=0.6\columnwidth]{random_noise_timeseries.pdf}
	\caption{Time series of random coincidences data...}
	\label{fig:random_coinciences}
\end{figure}

It was possible to characterise the noise here as we know it must follow a Poisson distribution. Using \verb|pymc3|, the noise was sampled to determine the mean of the Poisson distribution... The distribution of the random coincidences is shown in Figure~\ref{fig:random_coinciences_dist}

\begin{figure}[ht!]
	\centering
	\includegraphics[width=0.6\columnwidth]{random_noise_fitted_poisson.pdf}
	\caption{Distribution of random coincidences data... and Poisson distribution of the random coincidences, along with the median posterior fitted mean of the sample...}
	\label{fig:random_coinciences_dist}
\end{figure}

The median value of the posterior distribution, giving rise to the mean value of the Poisson distribution of random coincidence is $0.26$~counts/min.







%%%%%%%%%%%%%%%%%%%%%%%%%%%%%%%%%%%%%%%%%%%%%%%%%%%%%%%%%%%%%%%%%%%%%
%%%%%%%%%%%%%%%%%%%%%%%%%%%%%%%%%%%%%%%%%%%%%%%%%%%%%%%%%%%%%%%%%%%%%
\section{Conclusions}\label{sec:HS_14008_conclusion}
