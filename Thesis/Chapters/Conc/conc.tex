\chapter{Conclusions and Future Prospects}\label{chap:conc}

In this thesis, three studies have been presented, which investigated: using \gls{cr} detectors for space weather applications, the impact of solar activity on \glspl{gcr}, and solar interior-atmosphere linkages using observations the \gls{smmf}. %This work explores the interplay between the magnetic activity cycle and its effects as observed on the Earth.

In Chapter~\ref{chap:HiSPARC} we explored the properties and observations of the \gls{hisparc} experiment, to determine the feasibility of its use for monitoring space weather events. Using simulations of the interactions between \glspl{cr} and the Earth's magnetosphere, we were able to calculate the rigidity cut-off and \glspl{avd} of the \gls{hisparc} stations. We showed that the rigidity cut-off limits the observable \glspl{pcr} to those with energies on the order of and above $\sim 10^9$~eV. This highlighted that \gls{hisparc} stations would not observe the particles most susceptible to space weather events, i.e. those with energies $\sim 10^7-10^9$~eV.

We showed that we were unable to clearly detect signatures, in the raw \gls{hisparc} data, of the \glspl{fd} or \glspl{gle} that occurred over the lifetime of the \gls{hisparc} network. In addition, we showed that the effects of meteorological conditions caused variations in the data, which limited our ability to detect these events. A method to correct for the effects of atmospheric pressure and temperature was successfully demonstrated and applied to the \gls{hisparc} data. Following the correction of the atmospheric effects, the search for evidence of \glspl{gle} was repeated with the corrected \gls{hisparc} data. However, we concluded that we were also unable to clearly claim any observations of space weather events in the corrected \gls{hisparc} data.

\gls{as} simulations were employed to calculate the flux of muons at ground level. For low-energy \glspl{cr} ($\sim 10^9$~eV), we found the flux was small, producing very diffuse air showers of only a few muons, instead of the \glspl{eas} that \gls{hisparc} intends to observe. We showed the configuration of the \gls{hisparc} stations, which strongly relies on the triggering of multiple detectors within a station, biased observations to higher energy \glspl{pcr}, hence limiting the capabilities of using \gls{hisparc} network for space weather observations. This provided some explanation of why we were unable to detect the space weather events in the \gls{hisparc} data.

Furthermore, we ran simulations to predict the increase in the \gls{hisparc} count rate for some of the largest \glspl{gle} to-date. This showed that, on average, we expect an increase in the muon count rate using the \gls{hisparc} detectors of $<1\%$, for a `typical' \gls{gle}. It was shown that only the most energetic events, with a lower occurrence rate, would induce an increase in the \gls{hisparc} counts by $\gtrsim5\%$ and hence be potentially detectable. This provided further evidence to explain why we were unable to observe the \glspl{gle} and \glspl{fd} in the \gls{hisparc} data. Therefore, we concluded that the \gls{hisparc} network is generally incompatible with monitoring the lower limits of space weather activity, and only suitable as a monitor of the rarer and more extreme events.


Leading on from these results, in Chapter~\ref{chap:HiSPARC_14008} we presented an alternative \gls{hisparc} station configuration, with a novel arrangement of the detectors, and investigated its performance for use monitoring space weather events. Firstly, we outlined the configuration and technical set-up of the station; secondly, we performed the relevant atmospheric corrections, where we showed that the new station provides an accurate measure of the temperature inside the roof boxes for station 14008. Here, we also demonstrated that these temperature data were also an accurate measure for the detectors of station 14001, which are located on the same building, on the University of Birmingham campus.

Using a Bayesian method to sample from the posterior distribution, we found the mean count rate of the new station configuration was $\sim80~\upmu/\mathrm{s}$, which was in good agreement with the predicted values from the air shower simulations in Chapter~\ref{chap:HiSPARC}. Furthermore, we determined the noise from spurious counts is of about $0.0043\pm0.0002~\upmu/\mathrm{s}$, which is negligible compared to the Poisson noise representing $\sim11~\%$ of the signal. Comparing the data to that collected by a nearby \gls{nm} station, Dourbes---in Belgium, we showed that there was a good visual agreement between the two data sets. This relationship should be continually monitored, as it will be instrumental in the verification of the new station configuration when a space weather event occurs in the next Solar Cycle.

Simulations of artificial data were performed to assess the likelihood of observing \glspl{gle} in the new station configuration. We demonstrated that with 10-s cadence observations we expect to be able to detect \glspl{gle} with a magnitude of $\gtrsim3-4~\%$. Furthermore, through averaging the data into 1- or 5-minute bins, we showed this improved the sensitivity to observe \glspl{gle} with magnitudes $\gtrsim1.5-2.0~\%$. These values are in line with some of the predicted \gls{gle} magnitudes from Chapter~\ref{chap:HiSPARC}, providing compelling evidence to suggest that we should be capable of observing \glspl{gle} in this configuration.

We also simulated the performance of a network of detectors in this configuration and showed that we can improve the sensitivity to observe \glspl{gle} with magnitudes on the order of $\sim 1 \%$, through analysing the cross-correlation of nearby stations. However, we note that there is a strong dependence on decay time of the \glspl{gle}. We concluded that any upgrades to form a network of stations in this configuration should ideally use at least 5 stations, and 10 stations would be more beneficial.


\vspace{2em}


In Chapter~\ref{chap:GCR_SSN_24} we studied long-term variations of \gls{gcr} intensity in relation to the \gls{ssn} during the most recent solar cycles. We investigated the time lag between the \gls{gcr} intensity and the \gls{ssn}, and the hysteresis effect of the \gls{gcr} count rate against \gls{ssn} for Solar Cycles 20--24.

We showed that in cycle 24, the \gls{gcr} intensity lagged behind the \gls{ssn} by 2--4~months, which was slightly longer than the preceding even-numbered solar activity cycles (approx. 0--1~months). We showed the lag was not as large as the preceding odd-numbered cycles, and cycle 24 followed the trend of a short or near-zero lag for even-numbered cycles. We concluded that the cause of the extended lag in cycle 24 compared to previous even-numbered cycles was related to the deep, extended minimum between cycle 23 and 24, and the low maximum activity of cycle 24.

In addition, we showed the difference in the shapes of the hysteresis plots for odd-numbered and even-numbered cycles. The hysteresis plots were modelled using both a simple linear model and an ellipse model; the results showed that cycle 24 followed the same trend as preceding even-numbered cycles and was best represented by a straight line rather than an ellipse.

The time lag analysis was repeated using data from \gls{hisparc} station 501 (Nikhef). However, it was quantitatively concluded that there exists no correlation between the \gls{ssn} and the muon count rate measured by \gls{hisparc} station 501. The limiting factor to observe the effect was changes in set-up of the \gls{hisparc} station over time, which counteracted the expected variation due to solar activity.


\vspace{2em}


In Chapter~\ref{chap:SMMF}, a frequency-domain analysis of over 20~years of high-cadence \gls{bison} observations of the \gls{smmf} was presented. If we convert a time series of the \gls{smmf} to the frequency-domain, a strong \gls{rm} signal appears as a series of peaks. This characteristic demonstrates that the source of the \gls{smmf} is long-lived, over several rotations. The power spectrum of the \gls{bison} \gls{smmf} data was modelled to draw conclusions about the morphology of the \gls{smmf}, particularly focusing on the source of the rotationally modulated component in the signal.

The duty cycle for the 40-second cadence observations was very low, hence the effect of the low fill on the power spectrum of the \gls{smmf} was investigated to inform how to best model the complete power spectrum. This highlighted that although there appeared to exist a red-noise-like, stochastic background component in the power spectrum, this was a feature originating from power aliasing, due to the low duty cycle of the observations. We had to be very cautious in our approach when modelling the power spectrum to ensure that Parseval's theorem was obeyed and that the effects of the window function were robustly accounted for.

Using a Bayesian approach, a model was fitted to the power spectrum. We found that the \gls{rm} component had a frequency of $0.4270\pm0.0018\,\upmu\mathrm{Hz}$. This frequency allowed us to infer the sidereal period of the \gls{rm} signal to be $25.23\pm0.11$~days which suggested cycle-averaged latitude of $\sim 12^{\circ}$, thus linking the source to active bands of latitude on the Sun. From the width of the \gls{rm} component peak, we were able to determine the lifetime of its source. We measured the lifetime to be $139.6\pm18.5$~days, which is in the region of $\sim20\pm3$~weeks.

The measured properties of the \gls{rm} component of the \gls{smmf} were consistent with \glspl{ar}. The literature provided compelling arguments to suggest that sunspots were not the origin of the \gls{smmf}, therefore we concluded that, more generally, \glspl{ar} and \glspl{mfc} are the source of the dominant, rotation signal in the \gls{smmf}, that are long-lived on the solar disc and exist in active latitudes. In addition, we demonstrated, numerically and analytically, that our ability to determine the linewidth and hence lifetime of the \gls{rm} modes was unaffected by \gls{ar} migration and differential rotation.


In Chapter~\ref{chap:rmode} we further investigated the \gls{bison} \gls{smmf} data to search for evidence of a magnetic signature of global Rossby modes ($r$ modes) in the residual power spectrum. A well-resolved peak was identified near the predicted $l=2=m$ $r$ mode frequency. Using a Bayesian modelling technique we found the peak was centred of frequency of $550\pm19$~nHz (i.e. $\sim9.2$~nHz from the predicted frequency, but within measured uncertainty). In addition we measured the width of the peak to be $5.2^{+4.9}_{-2.8}$~nHz, and amplitude to be $\sim 27.1^{+7.9}_{-5.9}$~mG. The properties of the measured peak were in agreement with observations of other sectoral $r$ modes and in-line with the predictions for the $l=2=m$ $r$ mode.

To understand the way the $r$ mode would manifest in the power spectrum, due to the variation in the B$_0$ angle we generated simulated data and acquired additional  hemispheric observations of the \gls{smmf} using full-disc magnetograms. Through the analysis of these data, we showed how one might expect to see a prominent mode at the theoretical frequency, and not a split mode due to the effect of the B$_0$ variation. This further supported the hypothesis that the peak may have been the $l=2=m$ $r$ mode, assuming the magnetic $r$ mode signal had similar characteristics to the observed \gls{smmf} signal.

Finally, we investigated the power spectra of the \gls{smmf} observed with the \gls{wso} and \gls{sdo/hmi}, to verify if the peak was consistent across all observations. However, these observations showed no statistically significant peak in the location of the predicted $l=2=m$ $r$ mode frequency. It therefore made it highly unlikely that the candidate peak in the \gls{bison} spectrum was the $l=2=m$ $r$ mode. Because this peak was only observed in one of three data sets, and particularly not in the \gls{sdo/hmi} data which recent observations of sectoral Rossby waves in the Sun all used, we could not conclude that the candidate peak in the \gls{bison} spectrum was the $l=2=m$ $r$ mode. 


%%%%%%%%%%%%%%%%%%%%%%%%%%%%%%%%%%%%%%%%%%%%%%%%%%%%%%%%%%%%%%%%%%%%%
\subsection*{Future Prospects}

The first studies, presented in Chapter~\ref{chap:HiSPARC} and Chapter~\ref{chap:HiSPARC_14008}, of this thesis represented the beginning of exploring how the \gls{hisparc} network can be used to monitor space weather. The results from the first chapter showed that the \gls{hisparc} stations, in their original configurations, are not suitable for observing the effects of space weather; however, unfortunately, there have been few space weather events that have occurred over the lifetime of the \gls{hisparc} experiment and it is possible with more observations that we will observe a first space weather event with the existing \gls{hisparc} network. Continued monitoring of the data is imperative to check whether future space weather events are measured. This should be further supported by the observations from the \gls{gnmn} \citep{mishev_current_2020}, as they are proven to observe these events.

The air shower analysis to predict the variation in the muon count rate during space weather events was far from complete. We predicted the increase in muon intensity for only 6 of the 72 \glspl{gle} to-date. This was the limit of the analysis using the \gls{maire} tool used in this work. A more comprehensive analysis could be performed, using other sources of \gls{cr} spectra during space weather events, to predict the effect on the muon flux of the other \glspl{gle} and also \glspl{fd}.

The new station configuration shows promise to improve the space weather capabilities of the \gls{hisparc} network and it could be the beginning of a network-wide reconfiguration. The data collected by \gls{hisparc} station 14008 should be maintained until at least 2026, to ensure a complete study is performed up to the maximum of Solar Cycle 25 \citep{mcintosh_overlapping_2020, pesnell_lessons_2020}, when it is more probable that space weather events will occur. However, it would be of significant benefit to support this configuration for as long as possible, to compare the performance against the original \gls{hisparc} configuration for detecting space weather events.

The work presented in this chapter also investigated the effects of building a network of stations using this new configuration. It would be timely to create this network in the near-term, before cycle 25 maximum, to benefit from the improved sensitivity at the time when more space weather events are expected.


\vspace{2em}


It is also of interest to revisit \citet{ross_behaviour_2019} (i.e. the work presented in Chapter~\ref{chap:GCR_SSN_24}) to: (i) confirm the conclusions are the same up to the end of cycle 24/start of cycle 25; (ii) investigate the properties of cycle 25; (iii) analyse consistent, good-quality data from the \gls{hisparc} network, in both the original configuration and station 14008 configuration, to observe the solar cycle modulation of \glspl{gcr}.


\vspace{2em}


It would be highly desirable to improve the magnetic field capabilities of \gls{bison} to increase the duty cycle of the data and reduce the effect of the window function. In addition, with more observations, the frequency resolution will improve, allowing for more accurate inferences on the \gls{smmf} morphology. Plans were in place to re-acquire observations of the \gls{smmf} using the Sutherland node of \gls{bison} before the COVID-19 pandemic struck, which delayed the work. This will be done when it is possible to travel again. It would also be advantageous to provide this capability elsewhere in the network.

%Further to this, it would be interesting to investigate whether there exists a solar cycle dependence on the properties of the fitted peak. To do this, one could combine the data into separate maximum and minimum activity data sets, and re-run the analysis to determine if there are any differences. It would also be useful to do the same for rising and falling activity, to investigate the differences in the \gls{smmf} properties.

%In addition, it would be beneficial to revisit some of the magnetogram thresholding techniques that are used in the literature, to pin-point which specific phenomena associated with \glspl{ar} cause the \gls{smmf}. We have shown that the source is manifested in long-lived \glspl{ar}, in active latitudes; probing this further would allow us to infer more information on morphology of the \gls{smmf}.

Finally, on the future prospects of the suspected Rossby mode in the \gls{bison} data. Again, as we collect more observations of the \gls{smmf} using \gls{bison}, the frequency resolution of the power spectrum improves. An obvious next step in this work is to collect more observations of the \gls{smmf} with \gls{bison}, to further investigate if this suspected mode remains resolved, or whether it diminishes into the noise. If still resolved, the investigation should be repeated, to determine the source of the signal.